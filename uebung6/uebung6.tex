\documentclass{article}
\usepackage{listings} \lstset{numbers=left, numbersep=5pt}
%\lstset{language=Perl} 

\oddsidemargin0mm
\evensidemargin0mm
\topmargin-20mm
\textwidth160mm
\textheight240mm
\parindent0pt

\newcommand{\C}{\mathbb{C}}
\newcommand{\N}{\mathbb{N}}
\newcommand{\Z}{\mathbb{Z}}
\newcommand{\Q}{\mathbb{Q}}
\newcommand{\R}{\mathbb{R}}
\newcommand{\K}{\mathbb{K}}
\usepackage[ngerman]{babel}   % provide non-american language - new german
\usepackage[ansinew]{inputenc} % nur fuer schoene Umlaute ;)
\usepackage{lscape}
\usepackage{multirow}
\usepackage{expdlist}
%\usepackage{bigstrut} % \bigstrut in Tabellenzeilen, deren hochgestellte Eintraege von der \hline drueber durchgestrichen werden
\usepackage{amsmath, amsthm, amssymb}
\usepackage{stmaryrd}
\usepackage{fancyhdr}
\usepackage{graphicx}
\usepackage{listings}
\newcommand{\serie}{6}

\pagestyle{fancy}
\fancyhf{}
\fancyhead[L]{Sebastian D\"orner (180766)} %Kopfzeile links
\fancyhead[C]{Algorithmische Geometrie} %zentrierte Kopfzeile
\fancyhead[R]{Mittwoch} %Kopfzeile rechts
\renewcommand{\headrulewidth}{0.0pt} %obere Trennlinie
\fancyfoot[C]{\thepage} %Seitennummer

\begin{document}
\begin{large}
\textbf{L"osung \"Ubung \serie}\\ \\
\end{large}
\textbf{Aufgabe 2}\\
Ich nehme an, dass die Eckpunkte der Polygone bereits sortiert vorliegen. Ansonsten k\"onnten wir mit $n_2=0$ n\"amlich in $O(n_1)$ sortieren (vgl. Aufgabe 2.1).
Pseudocode unter Vernachl\"assung der Betrachtung von Randf\"allen:

\begin{minipage}{.6\textwidth}
\texttt{01 $p\ \leftarrow$ findMinimum($P_1$)\\
02 $q\ \leftarrow$ findMinimum($P_2$)\\
03 $p_{1}\ \leftarrow$ $p$\\
04 $p_{2}\ \leftarrow$ $p$\\
05 $q_{1}\ \leftarrow$ $q$\\
06 $q_{2}\ \leftarrow$ $q$\\
07 searchSet $\leftarrow \{p_1,p_2,q_1,q_2\}$\\
08 {\bf while} (searchSet $\neq \emptyset$)\\
09 $\quad$ min = minimum(searchSet)\\
10 $\quad$ report min\\
//setze entsprechenden Zeiger aus searchSet weiter\\
11 $\quad$ min.increaseOriginalPointer()\\
12 $\quad$ {\bf if} ($p_1.prev() = p_2$)\\
13 $\quad\quad$ searchSet $\leftarrow$ searchSet$\setminus \{p_1,p_2\}$\\
14 $\quad$ {\bf if} ($q_1.prev() = q_2$)\\
15 $\quad\quad$ searchSet $\leftarrow$ searchSet$\setminus \{q_1,q_2\}$
}
\end{minipage}
\begin{minipage}{.4\textwidth}
Mittels der Zeiger $p_1,p_2,q_1,q_2$ bewegen wir uns um die Polygone. Dabei nutzen wir die Methode increaseOriginalPointer(), um Zeiger mit Index 1 zu erh\"ohen und Zeiger mit Index 2 zu verringern. Wegen der Konvexit\"at muss das n\"achste Minimum jeweils unter diesen 4 Punkten sein. Alle einzelnen Anweisungen au\ss er Zeile 07 haben Laufzeit $O(1)$. Die while-Schleife wird $O(n_1 + n_2)$ mal ausgef\"uhrt, da sich die Zeiger $p_1$ und $p_2$ dann treffen, wenn wir ein mal das gesamte Polygon P umlaufen haben. Analoges gilt f\"ur $q_1$ und $q_2$.
\end{minipage}
\end{document}
